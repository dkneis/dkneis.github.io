\documentclass[10pt,a4paper]{article}
\usepackage[utf8]{inputenc}
\usepackage{graphicx}
\usepackage{url}
\author{dkneis}
\begin{document}
\newcommand{\software}[1]{\texttt{#1}}

{\LARGE How to run the \software{rodeo} examples}

\vspace{5mm}
Updated \today{} by \url{david.kneis@tu-dresden.de}

\vspace{5mm}
\hrule

\tableofcontents

\vspace{5mm}
\hrule

\section{Installation of \software{R} and auxiliary tools} \label{sec:install}

Extensive help on the installation of R and related utilities can be found on \url{https://cran.r-project.org/doc/manuals/r-release/R-admin.html}. Please consult this document if the following short instructions are insufficient.

\subsection{Install \software{R}}
Make sure that you really need to install \software{R}. If this is the case (first install or upgrade from an old version), go to \url{https://cloud.r-project.org/} and follow the download links in the box at the top of the page. Linux users probably want to use a package manager instead.

\subsection{Install auxiliary tools}
Although \software{rodeo} is not a so-called \emph{source package}, it requires the tools that are used to build those packages. This is because of \software{rodeo}'s built-in Fortran code generator. Before installing any tools, check whether this is necessary. For example, you could enter the command

\begin{verbatim}
gfortran --version
\end{verbatim}

at a terminal. If this succeeds, i.e. shows a version info, the required tools may already be there. Otherwise, see below:

On a Linux system, one typically needs to install the GNU compiler collection, including \software{gfortran}. See the section 'Essential and useful other programs under a Unix-alike' on the web page mentioned at the top of Sec.~\ref{sec:install}.

Windows users need to install the so-called \software{Rtools} from \url{https://cran.r-project.org/bin/windows/Rtools}. Chose the version that is compatible with the installed \software{R} version. Please read the section 'The-Windows-toolset' (currently appendix D) on the web page mentioned at the top of Sec.~\ref{sec:install} to circumvent typical pitfalls during and after installation. I recommend to install into a directory whose name does not contain blanks. If the automatic installer has an option to edit the PATH environment variable, let it do so. 

\section{After the installation}

On Windows, make sure that the directory with the \software{Rtools} is listed at the very beginning of your PATH environment variable. The installation folder of \software{R} itself should also be included in PATH.

The two commands

\begin{verbatim}
gfortran --help
R --help
\end{verbatim}

should work now when entered in a shell.

\section{Installation of \software{R}-packages}

The following packages are required to run the examples:

\bigskip
\begin{tabular}{ll}
\software{rodeo} & The code generator \\
\software{deSolve} & Numerical solvers for differential equations \\
\software{readxl} & Reads spreadsheet data (.xlsx files) \\
\software{lhs} & Latin hypercube sampling methods \\
\end{tabular}

\bigskip
The packages are all available on CRAN (\url{https://cran.r-project.org/}) and installation is most conveniently done from within \software{R}, using

\begin{verbatim}
install.packages(c("rodeo", "deSolve", "readxl", "lhs"))
\end{verbatim}

The installation may take some time because dependent packages are installed along with the above-mentioned ones.

\end{document}
