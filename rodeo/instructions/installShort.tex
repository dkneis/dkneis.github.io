\documentclass[10pt,a4paper]{article}
\usepackage[utf8]{inputenc}
\usepackage{graphicx}
\usepackage{url}

\usepackage{hyperref}
\hypersetup{
    colorlinks=true, % make the links colored
    linkcolor=blue, % color TOC links in blue
    urlcolor=red, % color URLs in red
    linktoc=all % 'all' will create links for everything in the TOC
}

\author{dkneis}
\begin{document}
\newcommand{\software}[1]{\texttt{#1}}
\newcommand{\radmin}{\url{https://cran.r-project.org/doc/manuals/r-release/R-admin.html}}


{\LARGE How to run the \software{rodeo} examples}

\vspace{5mm}
Updated \today{} by \url{david.kneis@tu-dresden.de}

\vspace{5mm}
\hrule

\tableofcontents

\vspace{5mm}
\hrule

\section{Installation of \software{R} and auxiliary tools} \label{sec:install}

Extensive help on the installation of R and related utilities can be found on \radmin. Please consult this document if the following short instructions are insufficient.

\subsection{Install \software{R}}
Make sure that you really need to install \software{R}. If this is the case (first install or upgrade from an old version), go to \url{https://cloud.r-project.org/} and follow the download links in the box at the top of the page. Linux users probably want to use a package manager instead.

\subsection{Install auxiliary tools}
Although the \software{rodeo} package itself does not need compilation, it requires the respective compile/build tools when it is used. This is because of \software{rodeo}'s built-in Fortran code generator. Before installing any tools, check whether this is necessary. For example, you could enter the command

\begin{verbatim}
gfortran --version
\end{verbatim}

at a terminal. If this succeeds, i.e. shows a version info, the required tools may already be there. Otherwise, see below:

On a Linux system, one typically needs to install the GNU compiler collection, including \software{gfortran}. See the section 'Essential and useful other programs under a Unix-alike' on \radmin.

Windows users need to install the so-called \software{Rtools} from \url{https://cran.r-project.org/bin/windows/Rtools}. Chose the version that is compatible with the installed \software{R} version. Please read the section 'The-Windows-toolset' (currently appendix D) on \radmin{} to circumvent typical pitfalls during and after installation. In particular, I recommend to
\begin{itemize}
  \item install into a directory whose name does not contain blanks.
  \item let the automatic installer edit the PATH environment variable (if the option is there). 
\end{itemize}

\section{After the installation} \label{sec:postInstall}

On Windows, make sure that the directories where \software{R} and the \software{Rtools}  were installed are \textbf{actually} included in your PATH environment variable. In addition, the order in which the directories are listed is \textbf{essential} as pointed out in section 'The-Windows-toolset' of \radmin{} (currently appendix D). For example, on a Windows 8 system, the first 3 entries of the PATH variable should be (in that order):

\begin{verbatim}
C:\myPrograms\Rtools\bin
C:\myPrograms\Rtools\gcc-4.6.3\bin
C:\myPrograms\R\R-3.2.5\bin
\end{verbatim}

\noindent assuming that you both \software{R} and \software{Rtools} were installed to a custom folder \verb|myPrograms|. The shown version numbers are/were up-to-date on 2016-04-26.

There are several ways to check the contents of the PATH variable on Windows:

\begin{itemize}
  \item type \verb|Sys.getenv("PATH")| at the \software{R} prompt.
  \item type \verb|echo %PATH%| into a \software{CMD} terminal (the \software{DOS}-like black box).
  \item navigate to the jungle of control settings until you hopefully find the menu item where you can view/edit environment variables. The place differs between versions of Windows.
\end{itemize}

In order to \textbf{permanently} edit the PATH variable, you probably need to find and use the respective menu. Note that changes to the PATH variable will not instantly be visible/active in other programs (e.g. \software{R}, \software{CMD}). Before you can see/use the altered PATH settings, you need to restart those programs.

\medskip
\noindent If everything was set up properly, the two commands

\begin{verbatim}
R CMD SHLIB --help
gfortran --help
\end{verbatim}

\noindent should show some usage info when entered in a \software{CMD} terminal on Windows (or \software{bash} on Linux).

\section{Installation of \software{R}-packages}

The following packages are required to run the examples:

\bigskip
\begin{tabular}{ll}
\software{rodeo} & The code generator \\
\software{deSolve} & Numerical solvers for differential equations \\
\software{readxl} & Reads spreadsheet data (.xlsx files) \\
\software{lhs} & Latin hypercube sampling methods \\
\end{tabular}

\bigskip
The packages are all available on CRAN (\url{https://cran.r-project.org/}) and installation is most conveniently done from within \software{R}, using

\begin{verbatim}
install.packages(c("rodeo", "deSolve", "readxl", "lhs"))
\end{verbatim}

The installation may take some time because dependent packages are installed along with the above-mentioned ones.

The latest development version of \software{rodeo} can also be installed directly from the source code repository using the \software{devtools} package.

\begin{verbatim}
library("devtools")
install_github("dkneis/rodeo")
\end{verbatim}

\end{document}
